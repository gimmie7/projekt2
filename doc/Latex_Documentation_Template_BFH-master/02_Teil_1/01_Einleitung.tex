\chapter{Einleitung}
\label{chap:teil1_einleitung}

Ein intelligentes Z�hlersystem, das einzelne Ger�te aus dem Gesamtstromverbrauch erkennen kann -- dieser anspruchsvollen Aufgabe widmet sich diese Projekt 2 Arbeit. Der zugrunde liegende Gedanke des Projektes ist es, dass jedes einzelne Ger�t Strom und Spannung in unterschiedlicher Art und Weise beeinflusst und damit eine Art "` Signatur "'  im Stromnetz hinterl�sst. Diese Signatur wird als aggregierter Gesamtstromverbrauch von einem Smart Meter (EM340) erfasst. Mittels Neuronalen Netzen, eventbasierten Algorithmen oder schon mittels einfacher Kombinatorik k�nnen Muster im Gesamtstromverbrauch erkannt und einzelnen Ger�ten zugeordnet werden. Im Gegensatz zum Submetering Verfahren, welches mit mehreren Z�hlern arbeitet, identifiziert dieses System mit einem einzigen Z�hler die aktiven Ger�te, welche an das Stromnetz angeschlossen sind. Damit werden nicht nur Kosten f�r die Messhardware sowie Installation und Wartungsaufw�nde, sondern auch Stromkosten gespart. Denn die eigenen Stromkosten senken kann nur derjenige wirklich effektiv, der seinen Stromverbrauch kennt und genau weiss, bei welchen Ger�ten er bzw. sie mit dem Sparen ansetzen kann. Somit erm�glicht dieses System auch die verursachungsgerechte Aufteilung der anfallenden Energiekosten. Das bedeutet, dass sich jedem Ger�t exakt die Energiekosten zuordnen lassen, die f�r den Betrieb des Ger�tes angefallen sind. Dies mag sich f�r Privathaushalte m�glicherweise nicht interessant anh�ren, in der Wirtschaft dagegen umso mehr.   

\chapter{Motivation / Ziele}
\label{chap:teil1_motziel}

% Eintr�ge im Verzeichnis erscheinen lassen ohne hier eine Referenz einzuf�gen
\nocite{raichle:bibtex_programmierung}
\nocite{MiKTeX}
\nocite{KOMA}
\nocite{TeXnicCenter}










