\chapter{Einleitung}
\label{chap:teil1_einleitung}

Ein intelligentes Z�hlersystem, das einzelne Ger�te aus dem Gesamtstromverbrauch erkennen kann -- dieser anspruchsvollen Aufgabe widmet sich diese Projekt 2 Arbeit. Der zugrunde liegende Gedanke des Projektes ist es, dass jedes einzelne Ger�t Strom und Spannung in unterschiedlicher Art und Weise beeinflusst und damit eine Art "` Signatur "'  im Stromnetz hinterl�sst. Diese Signatur wird als aggregierter Gesamtstromverbrauch von einem Smart Meter (EM340) erfasst. Mittels Neuronalen Netzen, eventbasierten Algorithmen oder schon mittels einfacher Kombinatorik k�nnen Muster im Gesamtstromverbrauch erkannt und einzelnen Ger�ten zugeordnet werden. Im Gegensatz zum Submetering Verfahren, welches mit mehreren Z�hlern arbeitet, identifiziert dieses System mit einem einzigen Z�hler die aktiven Ger�te, welche an das Stromnetz angeschlossen sind. Damit werden nicht nur Kosten f�r die Messhardware sowie Installation und Wartungsaufw�nde, sondern auch Stromkosten gespart. Denn die eigenen Stromkosten senken kann nur derjenige wirklich effektiv, der seinen Stromverbrauch kennt und genau weiss, bei welchen Ger�ten er bzw. sie mit dem Sparen ansetzen kann. Somit erm�glicht dieses System auch die verursachungsgerechte Aufteilung der anfallenden Energiekosten. Das bedeutet, dass sich jedem Ger�t exakt die Energiekosten zuordnen lassen, die f�r den Betrieb des Ger�tes angefallen sind. Dies mag sich f�r Privathaushalte m�glicherweise nicht interessant anh�ren, in der Wirtschaft dagegen umso mehr.   

\chapter{Motivation / Ziele}
\label{chap:teil1_motziel}
Die Nachfrage nach Energie w�chst weltweit rasant, und die Nachfrage nach elektrischer Energie w�chst noch schneller [1]. Die Nachfrage nach elektrischer Energie wird sich bis zum Jahre 2050 voraussichtlich verdoppeln [2]. Um diese Energieprobleme zu l�sen, ist es notwendig, Elektrizit�t wirtschaftlch und intelligent zu nutzen. Ein Ansatz zur Steigerung der Effizienz des Stromverbrauchs besteht darin, positive Verhaltens�nderungen bei den Verbrauchern anzuregen. Dies kann durch Analysen des Energieverbrauchs erreicht werden. Non-Intrusive Load Monitoring (NILM) ist eine geeignete Methode dazu. NILM ist ein Prozess zur Analyse von �nderungen der Spannung und des Stroms, die in ein Haus gehen, und leitet daraus ab, welche Ger�te im Haus mit ihrem individuellen Energieverbrauch verwendet werden.

Das prim�re Ziel dieser Projekt 2 Arbeit ist es erstmal, sich mit dem ganzen Thema rund um NILM vertraut zu machen. Dazu geh�rt auch sich die elektrotechnischen Grundlagen zu erarbeiten, welche als Basis zur Signaturerstellung der Ger�te bzw. Verbraucher dienen. Denn um einen Energieverbrauch basierend auf NILM zu disaggregieren, muss man die Funktionsweise der Ger�te verstehen.
Die Disaggregation soll in einem ersten Schritt mit einfacher Kombinatorik implementiert werden. In einem n�chsten Schritt wird ein eventbasierter Ansatz angewendet. Als Zusatz soll dann das Disaggregationsverfahren noch im Hinblick auf Machine Learning und Neuronale Netze untersucht werden.

% Eintr�ge im Verzeichnis erscheinen lassen ohne hier eine Referenz einzuf�gen
\nocite{raichle:bibtex_programmierung}
\nocite{MiKTeX}
\nocite{KOMA}
\nocite{TeXnicCenter}










